\section{Initial Situation and Classification in the Existing Research}
The thesis can be classified into the scientific areas of interpreter construction and full-text search.\\
The topic of full-text search, with possibilities such as word and phrase-based search and the ability to index documents in their native form, such as office documents and pdfs, is an increasingly important feature of modern databases and document management systems and thus of great relevance. Making such a powerful tool, through easy-to-use search engines and grammar, as easily accessible as possible, without requiring prior knowledge of SQL, allows it to be used by individuals and companies with little IT knowledge. Just one example of why domain-specific languages and interpreters play an important role in today's IT world.\\
There are already many examples of how such custom languages and interpreters make writing code more accessible or support experienced software engineers in their development. Examples in the context of full-text search are search engines, which generate SQL statements from search queries.
\section{Approach and Methodology}
To achieve the objective of my thesis I will refer to literature that defines concepts of full-text search and the development of interpreters. Based on this I use the method of prototypical implementation and try to apply these concepts practically.
\section{Outline}
\begin{enumerate}
    \item Introduction
    \item Theory
    \begin{enumerate}
        \item Elaboration of the language elements for full-text search in common relational databases (Microsoft SQL Server)
        \item How to build interpreters
    \end{enumerate}
    \item Approach
    \begin{enumerate}
        \item Definition of grammar for full-text search input and schematic description using the extended Backus-Naur form
        \item Conversion of the grammar into a software component in a web application
    \end{enumerate}
    \item Source Code
    \item Examples and Usage
    \item Conclusion
\end{enumerate}