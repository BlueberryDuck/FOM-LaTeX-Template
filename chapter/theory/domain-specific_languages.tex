\subsection{Domain-Specific Languages}\label{sec:dsl}
Commonly known programming languages, such as C or Java, are also called a \ac{GPL}. \ac{GPL}s are designed to handle any problem with relatively equal levels of efficiency and expressiveness. However, many applications do not require a multifunctional \ac{GPL} and can describe a problem more naturally using a \ac{DSL}. \ac{DSL}s are languages that have been developed specifically for a particular application or domain, to be able to develop faster and more effectively \parencite[cf.][p. 1]{hudak_domain-specific_1997}. By tailoring notations and constructs to the domain in question, \ac{DSL}s offer significant gains in expressiveness and usability compared to \ac{GPL}s for the domain in question, with corresponding productivity gains and lower maintenance costs \parencite[cf.][p. 317]{mernik_when_2005}. \ac{DSL}s are by no means a product of modern software development but have existed since the beginning of programming. One of the first \ac{DSL}s ever designed was \ac{APT}, which was used for the development of numerically controlled machine tools in 1957 \parencite[cf.][pp. 283-284]{ross_origins_1978}.\\
\ac{DSL}s can be found everywhere in the world of IT, for example, this thesis was written with the help of \LaTeX{} to design layout and formatting. Table \ref{tbl:popular_dsl} lists some well-known \ac{DSL}s and their application/domain to give examples of what is classified as a \ac{DSL}.
\begin{table}[H]
    \caption{Popular DSLs}
    \label{tbl:popular_dsl}
    \begin{tabularx}{\textwidth}[ht]{|l|X|l|}
        \hline
        \textbf{DSL}         & \textbf{Applicaiton}                        \\
        \hline
        Lex and Yacc         & program lexing and parsing                  \\
        PERL                 & text/file manipulation/scripting            \\
        VDL                  & hardware description                        \\
        \TeX, \LaTeX, troff  & document layout                             \\
        HTML, SGML           & document markup                             \\
        SQL, LDL, QUEL       & databases                                   \\
        pic, postscript      & 2D graphics                                 \\
        Open GL              & high-level 3D graphics                      \\
        Tcl, Tk              & GUI scripting                               \\
        Mathematica, Maple   & symbolic computation                        \\
        AutoLisp/AutoCAD     & computer aided design                       \\
        Csh                  & OS scripting (Unix)                         \\
        IDL                  & component technology (COM/CORBA)            \\
        Emacs Lisp           & text editing                                \\
        Prolog               & logic                                       \\
        Visual Basic         & scripting and more                          \\
        Excel Macro Language & spreadsheets and many things never intended \\
        \hline
    \end{tabularx} \\
    \cite[Source:][p. 3]{hudak_domain-specific_1997}
\end{table}
Programs written in a \ac{DSL} are considered to be more concise, quicker to write, easier to maintain and easier to reason about and most importantly they can be written by non-programmers. In particular, experts in the domain for which the \ac{DSL} was developed can use \ac{DSL}s to program applications without having to acquire programming skills. An expert of a domain already knows the semantics of the domain, all that is needed to start development is the corresponding notation that expresses these semantics \parencite[cf.][pp. 2-4]{hudak_domain-specific_1997}.
