\newpage
\section{Conclusion}
\subsection{Summary}
The implementation can be summed up in a technical summary, highlighting milestones and their placement in the overall project.\\
First, the language was defined and the rough structure of the language was discussed. The query language should consist of functions starting with '@', which contain parameters. Afterward, it was defined which functions should be implemented, and which parameters are needed or allowed. The finished syntax definition was then written down in \ac{EBNF}.\\
The core of the prototype, the code generator itself, was divided into three modules: lexer, parser, and generator. The lexer analyzes the input string and cuts one or more characters into accepted tokens of the query language. The definition of these acceptable tokens is done by literal strings and regular expressions, whereas whitespaces are ignored by the lexer.\\
The list of tokens is passed to the parser, which goes linearly through the entire list and tries to verify the logical order and convert it into an ordered structure, the \ac{AST}. Statements are interpreted as functions, which contain expressions as search parameters. Special cases are operators, which are more complex and therefore treated in more detail in chapter \ref{sec:operators}. The use of operators requires the introduction of precedence, to build hierarchical structures, despite the linear processing of tokens. Thus also the final elements of the parser were implemented.\\
The final \ac{TSQL} is then created from the \ac{AST} using the generator. Parts of the \ac{TSQL} are preformulated by constants and the built-in function CONTAINSTABLE is used. The specifications of this function are filled with the contents of the \ac{AST}. Each of the implemented functions is translated to \ac{TSQL} in its way. In addition, the correct notation of operators is taken care of as well. All parts are combined into one long string and output as the result of the generator, and thus the whole code generator.\\
As part of the prototype, a website was also built as a user interface, where the user enters the query language request and results are displayed. In addition, other implemented functions execute the generated \ac{TSQL} and parse the results of the \ac{SQL} server.