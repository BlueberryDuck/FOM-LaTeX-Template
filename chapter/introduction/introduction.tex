\section{Introduction}
Current relational databases offer much more functionality than just transactional processing of data. One of these functionalities is the full-text search in documents, which allows possibilities like word and phrase-based searches and inflectional searches using specialized functions and combinations of query terms. In today's IT world, code generators, in conjunction with a graphical user interface or \ac{DSL}, make powerful tools such as full-text search more accessible, enabling individuals and companies with little IT knowledge to use them.\\
The development of a code generator including its \ac{DSL} can be allocated in the research fields of compiler construction and programming language design. Knowledge in these areas allows a more abstract view of many other areas of development and how superficially independent topics can find an application in IT. The question to be answered within the scope of the thesis is 'To what extent can a software component be developed using a custom query language and code generator to enable full-text search on \ac{SQL} servers?'\\
The objective is a code generator that uses a custom query language to define search criteria that enable full-text search. The code generator takes text as input and generates \ac{SQL} code which is executed on an \ac{SQL} server. The respective search results should be displayed to the user, without them having to interact with \ac{SQL} code or servers.\\
To achieve this objective, a prototype is developed that practically implements the concepts of compiler construction and represents the functionality more effectively than textual descriptions or purely static models. The implementation of a prototype is an iterative process: Implementation is followed by evaluating and adapting to new problems and specifications. In each case, as much functionality is implemented as is necessary to verify the targeted phase result \parencite[cf.][p. 3]{pomberger_methoden_1992}.